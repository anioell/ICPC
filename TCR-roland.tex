\documentclass[9pt]{article}   	% use "amsart" instead of "article" for AMSLaTeX format
\usepackage{geometry}                		% See geometry.pdf to learn the layout options. There are lots.
\geometry{letterpaper}                   		% ... or a4paper or a5paper or ... 
%\geometry{landscape}                		% Activate for for rotated page geometry
%\usepackage[parfill]{parskip}    		% Activate to begin paragraphs with an empty line rather than an indent
\usepackage{graphicx}				% Use pdf, png, jpg, or eps§ with pdflatex; use eps in DVI mode
								% TeX will automatically convert eps --> pdf in pdflatex		
\usepackage{amssymb}
\usepackage{amsmath}

\usepackage{color}

\definecolor{javared}{rgb}{0.6,0,0} % for strings
\definecolor{javagreen}{rgb}{0.25,0.5,0.35} % comments
\definecolor{javapurple}{rgb}{0.5,0,0.35} % keywords
\definecolor{javadocblue}{rgb}{0.25,0.35,0.75} % javadoc
 
\usepackage{listings}
\lstset{language=Java,
basicstyle=\ttfamily\scriptsize,
keywordstyle=\color{javapurple}\bfseries,
stringstyle=\color{javared},
commentstyle=\color{javagreen},
morecomment=[s][\color{javadocblue}]{/**}{*/},
numbers=left,
numberstyle=\tiny\color{black},
numbersep=10pt,
tabsize=2,
showspaces=false,
showstringspaces=false}

\title{NWERC TCR 2015}

\date{\today}							% Activate to display a given date or no date

\begin{document}
% \maketitle

\section{Mathematics}

\subsection{Divisibility}

\subsubsection{Algorithm}

\begin{lstlisting}
// calculates (alternating) k-digitSum for integer number given by M
public static long digit_sum(String M, int k, boolean alt) {
	long dig_sum = 0;
	int vz = 1;
	while (M.length() > k) {
		if (alt) vz *= -1;
		dig_sum += vz*Integer.parseInt(M.substring(M.length()-k));
		M = M.substring(0, M.length()-k);
	}
	if (alt) vz *= -1;
	dig_sum += vz*Integer.parseInt(M);
	return dig_sum;
}

// example: divisibility of M by 13
public static boolean divisible13(String M) {
	return digit_sum(M, 3, true)%13 == 0;
}
\end{lstlisting}

\subsubsection{Explanation}

$D \mid M \Leftrightarrow D \mid {\tt digit\_sum(M, k, alt)}$, refer to table for values of $D, k, alt$.

\begin{table}[htbp]
\begin{tabular}{| c || c | c | c | c | c | c | c | c | c | c |}
\hline
$D$ & 3 & 7 & 9 & 11 & 13 & 17 & 19 & 23 & 37 & 41 \\
\hline
$k$ & 1 & 3 & 1 & 2 & 3 & 8 & 9 & 11 & 3 & 5 \\
\hline
$alt$ & f & w & f & f & w & w & w & w & f & f \\ 
\hline
\end{tabular}
\end{table}

\subsection{Binomial Coefficient}

\subsubsection{Algorithms}
\begin{minipage}{.5\textwidth}
\begin{lstlisting}
// binomial coefficient for all K <= N
public static long[][] binomial_matrix(int N, int K) {
	long[][] B = new long[N+1][K+1];
	for (int k = 1; k <= K; k++) { 
		B[0][k] = 0;
	}
	for (int m = 0; m <= N; m++) {
		B[m][0] = 1;
	}
	for (int m = 1; m <= N; m++) {
		for (int k = 1; k <= K; k++) {
			B[m][k] = B[m-1][k-1] + B[m-1][k];
		}
	}
	return B;
}
\end{lstlisting}
Time Complexity: $\mathcal{O}(NK)$
\end{minipage}
\begin{minipage}{.5\textwidth}
\begin{lstlisting}	
// binomial coefficient (n choose k)
public static long bin(int n, int k) {
	if (k == 0) {
		return 1;
	} else if (k > n/2) {
		return bin(n, n-k);
	} else {
		return n*bin(n-1, k-1)/k;
	}
}
\end{lstlisting}
Time Complexity: $\mathcal{O}(k)$
\end{minipage}

\subsubsection{Properties}
$$ 1.\ {n \choose k} = \frac{n!}{k! (n-k)!} \quad 2.\ {n \choose k} = {n \choose n-k} \quad 3.\ {n+1 \choose k+1}={n \choose k}+{n \choose k+1} \quad 4.\ \sum\limits_{k = 0}^{n} {n \choose k} = 2^n$$

\subsection{Combinatorics}
\begin{itemize}
\item Variations (ordered): $k$ out of $n$ objects {\footnotesize (permutations for $k = n$)}
\begin{itemize}
	\item without repetition:\\ $M = \{ (x_{1}, \ldots , x_{k}): 1 \leq x_i \leq n, \ x_i \neq x_j \mbox{ if } i \neq j \}$, $|M|= \frac{n!}{(n-k)!}$
	\item with repetition:\\ $M = \{ (x_{1}, \ldots , x_{k}): 1 \leq x_i \leq n \}, |M| = n^k$
	\end{itemize}
\item Combinations (unordered): $k$ out of $n$ objects
\begin{itemize}
	\item without repetition: $M = \{ (x_{1}, \ldots , x_{n}): x_i \in \{0,1\}, \ x_1 + \ldots + x_n = k \}$, $|M| = {n \choose k}$
	\item with repetition: $M = \{ (x_{1}, \ldots , x_{n}): x_i \in \{0,1,\ldots,k\}, \ x_1 + \ldots + x_n = k \}$, $|M| = {n+k-1 \choose k}$
\end{itemize}
\item Ordered partition of numbers: $x_1+\ldots+x_k = n$ {\footnotesize (i.e. 1+3 = 3+1 = 4 are counted as 2 solutions)}
\begin{itemize}
	\item \#Solutions for $x_i \in \mathbb{N}_0$: ${n+k-1 \choose k-1}$
	\item \#Solutions for $x_i \in \mathbb{N}$: ${n-1 \choose k-1}$
\end{itemize}
\item Unordered partition of numbers: $x_1+\ldots+x_k = n$ {\footnotesize (i.e. 1+3 = 3+1 = 4 are counted as 1 solution)}
\begin{itemize}
	\item \#Solutions for $x_i \in \mathbb{N}$: $P_{n,k} = P_{n-k,k}+P_{n-1,k-1}$ where $P_{n,1} = P_{n,n} = 1$
\end{itemize}
\item Derangements {\footnotesize (permutations without fixed points)}: $!n = n! \sum\nolimits_{k = 0}^n \frac{(-1)^k}{k!} = \lfloor \frac{n!}{e} + \frac{1}{2} \rfloor$
\end{itemize}

\subsection{Polynomial Interpolation}
\subsubsection{Theory}
Problem: for $\{(x_0,y_0),\ldots, (x_n, y_n) \}$ find $p \in \Pi_{n}$ with $p(x_i) = y_i$ for all $i=0,\ldots,n$.\\
Solution: $p(x) = \sum\limits_{i=0}^n \gamma_{0,i} \prod\limits_{j=0}^{i-1} (x-x_i)$ where $\gamma_{j,k} = y_j$ for $k = 0$ and $\gamma_{j,k} = \frac { \gamma_{j+1,k-1} - \gamma_{j,k-1} }{x_{j+k}-x_j}$ otherwise. \\
Efficient evaluation of $p(x)$: $b_n = \gamma_{0,n}$, $b_i = b_{i+1}(x-x_i) + \gamma_{0,i}$ for $i=n-1,\ldots,0$ with $b_0 = p(x)$. \\

\subsubsection{Algorithms}
\begin{lstlisting}
public class interpol {
	
	// divided differences for points given by vectors x and y
	public static rat[] divDiff(rat[] x, rat[] y) {
		rat[] temp = y.clone();
		int n = x.length;
		rat[] res = new rat[n];
		res[0] = temp[0];
		for (int i=1; i < n; i++) {
			for (int j = 0; j < n-i; j++) {
				temp[j] = (temp[j+1].sub(temp[j])).div(x[j+i].sub(x[j]));
			}
			res[i] = temp[0];
		}
		return res;
	}
	
	// evaluates interpolating polynomial p at t for given 
	// x-coordinates and divided differences
	public static rat p(rat t, rat[] x, rat[] dD) {
		int n = x.length;
		rat p = new rat(0);
		for (int i = n-1; i > 0; i--) {
			p = (p.add(dD[i])).mult(t.sub(x[i-1]));
		}
		p = p.add(dD[0]);
		return p;
	}
	
	public static void main(String[] args) {
	
		rat[] test = {new rat(4,5), new rat(7,10), new rat(3,4)};
		test = rat.commonDenominator(test);
		for (int i = 0; i < test.length; i++) {
			System.out.println(test[i].toString());
		}
	
		rat[] x = {new rat(0),new rat(1), new rat(2), new rat(3), new rat(4), new rat(5)};
		rat[] y = {new rat(-10), new rat(9), new rat(0), new rat(1), new rat(1,2), new rat(1,80)};
		rat[] dD = divDiff(x,y);
		System.out.println("p("+7+") = "+p(new rat(7), x, dD));
	}

}
\end{lstlisting}
\begin{lstlisting}
// implementation of rational numbers
class rat {

	public long c;
	public long d;
	
	public rat (long c, long d) {
		this.c = c;
		this.d = d;
		this.shorten();
	}
	
	public rat (long c) {
		this.c = c;
		this.d = 1;
	}
	
	public static long ggT(long a, long b) {
		while (b != 0) {
			long h = a%b;
			a = b;
			b = h;
		}
		return a;
	}
	
	public static long kgV(long a, long b) {
		return a*b/ggT(a,b);
	}
	
	public static rat[] commonDenominator(rat[] c) {
		long kgV = 1;
		for (int i = 0; i < c.length; i++) {
			kgV = kgV(kgV, c[i].d);
		}
		for (int i = 0; i < c.length; i++) {
			c[i].c *= kgV/c[i].d;
			c[i].d *= kgV/c[i].d;
		}
		return c;
	}
	
	public void shorten() {
		long ggT = ggT(this.c, this.d);
		this.c = this.c / ggT;
		this.d = this.d / ggT;
		if (d < 0) {
			this.d *= -1;
			this.c *= -1;
		}
	}
	
	public String toString() {
		if (this.d == 1) return ""+c;
		return ""+c+"/"+d;
	}
	
	public rat mult(rat b) {
		return new rat(this.c*b.c, this.d*b.d);
	}
	
	public rat div(rat b) {
		return new rat(this.c*b.d, this.d*b.c);
	}
	
	public rat add(rat b) {
		long new_d = kgV(this.d, b.d);
		long new_c = this.c*(new_d/this.d) + b.c*(new_d/b.d);
		return new rat(new_c, new_d);
	}
	
	public rat sub(rat b) {
		return this.add(new rat(-b.c, b.d));
	}
	
}
\end{lstlisting}

\subsection{Fibonacci Sequence}
\subsubsection{Binet's formula}
$
\begin{pmatrix}
f_n \\
f_{n+1}
\end{pmatrix} =
\begin{pmatrix}
0 & 1 \\
1 & 1
\end{pmatrix}^n
\begin{pmatrix}
0 \\
1
\end{pmatrix}
\Rightarrow
f_n = \frac{1}{\sqrt{5}} (\phi^n - \tilde{\phi}^n)$ where $\phi = \frac{1+\sqrt{5}}{2}$ and $\tilde{\phi} = \frac{1-\sqrt{5}}{2}$.
\subsubsection{Generalization}
$g_n = \frac{1}{\sqrt{5}} (g_0 (\phi^{n-1} - \tilde{\phi}^{n-1}) + g_1 (\phi^n - \tilde{\phi}^n)) = g_0 f_{n-1} + g_1 f_n$ for all $g_0, g_1 \in \mathbb{N}_0$

\subsubsection{Pisano Period}
Both $(f_n \bmod k)_{n \in \mathbb{N}_0}$ and $(g_n \bmod k)_{n \in \mathbb{N}_0}$ are periodic.
\begin{table}[htbp]
\begin{tabular}{| c || c | c | c | c | c | c | c | c | c | c | c | }
\hline
$k$ & 2 & 3 & 4 & 5 & 6 & 7 & 8 & 9 & 10 & 100 & $10^n$ for $n >2$ \\
\hline
$\pi(k)$ & 3 & 8 & 6 & 20 & 24 & 16 & 12 & 24 & 60 & 300 & $15 \cdot 10^{n-1}$ \\
\hline
\end{tabular}
\end{table}
\end{document}  